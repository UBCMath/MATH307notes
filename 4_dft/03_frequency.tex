\section{Frequency, Amplitude and Phase}

\begin{bigidea}
The DFT of a signal computes the amplitude and phase of each frequency in the signal.
\end{bigidea}

\begin{definition}
The DFT is used to study sound, images and any kind of information that can be represented by a vector $\bs{x} \in \mathbf{C}^N$. Therefore, in the context of the DFT, we use the term {\bf signal} to refer to a vector $\bs{x} \in \mathbf{C}^N$ and we use the notation $\bs{x}[n] = x_n$ to refer to the entries
$$
\bs{x} = \begin{bmatrix} \ x_0 & x_1 & x_2 & \cdots & x_{N-1} \ \end{bmatrix}^T
$$
\end{definition}

\begin{definition}
Let $N$ be a positive integer and let
$$
\bs{n} = \begin{bmatrix} 0 & 1 & 2 & \cdots & N-1 \end{bmatrix}^T
\hspace{10mm}
\bs{t} = (1/N) \bs{n} = \begin{bmatrix} 0 & 1/N & 2/N & \cdots & (N-1)/N \end{bmatrix}^T
$$ 
A {\bf sinusoid} is a signal of the form
$$
\bs{x} = A \cos(2\pi k \bs{t} + \phi)
$$
where $k$ is the {\bf frequency} (in periods per $N$ samples), $A$ is the {\bf amplitude} and $\phi$ is the {\bf phase}. Here we use vector notation
$$
A \cos(2\pi k \bs{t} + \phi) = \begin{bmatrix} A \cos(\phi) \\ A \cos(2\pi k (1/N) + \phi) \\ A \cos(2\pi k (2/N) + \phi) \\ \vdots \\ A \cos(2\pi k (N-1)/N + \phi) \end{bmatrix}
$$
\end{definition}

\begin{example}
Let $N = 8$ and so
\begin{align*}
\bs{n} &= \begin{bmatrix} 0 & 1 & 2 & 3 & 4 & 5 & 6 & 7 \end{bmatrix}^T \\
\bs{t} &= \begin{bmatrix} 0 & 1/8 & 1/4 & 3/8 & 1/2 & 5/8 & 3/4 & 7/8 \end{bmatrix}^T
\end{align*}
Consider the signal
$$
\bs{x} = \cos(2 \pi \bs{t} ) = \begin{bmatrix} 1 & 1/\sqrt{2} & 0 & -1/\sqrt{2} & -1 & -1/\sqrt{2} & 0 & 1/\sqrt{2} \end{bmatrix}^T
$$
and sketch the signal as a stemplot
\begin{center}
\begin{tikzpicture}
\draw [->] (0,0) -- (8,0) node[below right] {$n$}; \draw [->] (0,-1.5) -- (0,1.5) node[above left] {$\bs{x}[n]$};
\draw (0.1,1) -- (-0.1,1) node[left] {1}; \draw (0.1,-1) -- (-0.1,-1) node[left] {$-1$};
\foreach \x/\y in { 0/1 , 1/0.7071 , 2/0 , 3/-0.7071 , 4/-1 , 5/-0.7071 , 6/0 , 7/0.7071} { \filldraw (\x,0) -- (\x,\y) circle[radius=2pt]; }
\foreach \k in {0,...,7} { \draw (\k,0) node[below right] {$\k$}; }
\end{tikzpicture}
\end{center}

Now consider the signal
$$
\bs{x} = \cos(4 \pi \bs{t} ) = \begin{bmatrix} 1 & 0 & -1 & 0 & 1 & 0 & -1 \end{bmatrix}^T
$$
and sketch the signal as a stemplot
\begin{center}
\begin{tikzpicture}
\draw [->] (0,0) -- (8,0) node[below right] {$n$}; \draw [->] (0,-1.5) -- (0,1.5) node[above left] {$\bs{x}[n]$};
\draw (0.1,1) -- (-0.1,1) node[left] {1}; \draw (0.1,-1) -- (-0.1,-1) node[left] {$-1$};
\foreach \x/\y in { 0/1 , 1/0 , 2/-1 , 3/0 , 4/1 , 5/0 , 6/-1 , 7/0} { \filldraw (\x,0) -- (\x,\y) circle[radius=2pt]; }
\foreach \k in {0,...,7} { \draw (\k,0) node[below right] {$\k$}; }
\end{tikzpicture}
\end{center}

\end{example}

\begin{proposition}
\begin{align*}
\bs{f}_k & = \cos(2 \pi k \bs{t}) + i \sin(2 \pi k \bs{t}) \\
\frac{1}{2} \left( \bs{f}_k + \overline{\bs{f}}_k \right) &= \cos(2 \pi k \bs{t}) \\
\frac{1}{2i} \left( \bs{f}_k - \overline{\bs{f}}_k \right) &= \sin(2 \pi k \bs{t})
\end{align*}

\begin{proof}
We showed in a previous proposition that
$$
\omega_N^k = \cos\left( \frac{2 \pi k}{N} \right) + i \sin \left( \frac{2 \pi k}{N} \right)
$$
therefore
$$
\bs{f}_k =
\renewcommand{\arraystretch}{1.5}
\begin{bmatrix} 1 \\ \omega^k_N \\ \omega^{2k}_N \\ \vdots \\ \omega^{(N-1)k}_N \end{bmatrix}
=
\begin{bmatrix} 1 \\ \cos(2\pi k (1/N)) + i \sin(2\pi k (1/N)) \\ \cos(2\pi k (2/N)) + i \sin(2\pi k (2/N)) \\ \vdots \\ \cos(2\pi k (N-1)/N) + i \sin(2\pi k (N-1)/N) \end{bmatrix}
= \cos(2 \pi k \bs{t}) + i \sin(2 \pi k \bs{t})
$$
Further,
\begin{align*}
\omega_N^{nk} + \omega_N^{-nk} &= \cos\left( \frac{2 \pi n k}{N} \right) + i \sin \left( \frac{2 \pi n k}{N} \right) + \cos\left( \frac{2 \pi n k}{N} \right) - i \sin \left( \frac{2 \pi n k}{N} \right) \\
&= 2\cos\left( \frac{2 \pi n k}{N} \right) 
\end{align*}
Therefore
$$
\bs{f}_k + \overline{\bs{f}}_k =
\begin{bmatrix} 1 \\ \omega^k_N \\ \omega^{2k}_N \\ \vdots \\ \omega^{(N-1)k}_N \end{bmatrix} +
\begin{bmatrix} 1 \\ \omega^{-k}_N \\ \omega^{-2k}_N \\ \vdots \\ \omega^{-(N-1)k}_N \end{bmatrix}
= \begin{bmatrix} 2 \\ 2\cos( 2 \pi k/N ) \\ 2\cos( 2 \pi n (2k)/N ) \\ \vdots \\ 2\cos( 2 \pi (N-1) k/N ) \end{bmatrix}
= 2 \cos(2 \pi k \bs{t})
\renewcommand{\arraystretch}{1}
$$
The last equality is proved similarly.
\end{proof}
\end{proposition}

\begin{proposition}
Let $\bs{x} = A \cos(2 \pi k \bs{t} + \phi)$. Then
$$
\mathrm{DFT}(\bs{x}) = \frac{AN}{2} e^{i \phi} \, \bs{e}_k + \frac{AN}{2} e^{-i \phi} \, \bs{e}_{N-k}
$$

\begin{proof}
We proved in a previous proposition
$$
\cos(2 \pi k \bs{t} ) = \frac{1}{2} \left( \bs{f}_k + \overline{\bs{f}}_k \right) = \frac{1}{2} \left( \bs{f}_k + \bs{f}_{N-k} \right)
$$
and we also showed that
$$
\mathrm{DFT}( \bs{f}_k ) = N \bs{e}_k
$$
Compute
$$
\mathrm{DFT}(\cos(2 \pi k \bs{t} )) = \frac{1}{2} \mathrm{DFT}(\left( \bs{f}_k + \bs{f}_{N-k} \right)) = \frac{1}{2} \left( N\bs{e}_k + N \bs{e}_{N-k} \right)
$$
Similarly, compute
$$
\mathrm{DFT}(\sin(2 \pi k \bs{t} )) = \frac{1}{2i} \mathrm{DFT}(\left( \bs{f}_k - \bs{f}_{N-k} \right)) = \frac{1}{2i} \left( N\bs{e}_k - N \bs{e}_{N-k} \right)
$$
Use the trigonometric identity
$$
\cos(\alpha + \beta) = \cos \alpha \cos \beta - \sin \alpha \sin \beta
$$
to find
\begin{align*}
\mathrm{DFT}(\bs{x}) &= \mathrm{DFT}(A \cos(2 \pi k \bs{t} + \phi)) \\
&= A \cos(\phi) \, \mathrm{DFT}(\cos(2 \pi k \bs{t}))  - A  \sin(\phi) \, \mathrm{DFT}(\sin(2 \pi k \bs{t})) \\
&= \frac{A \cos(\phi)}{2} \left( N\bs{e}_k + N \bs{e}_{N-k} \right) - \frac{A \sin(\phi)}{2i} \left( N\bs{e}_k - N \bs{e}_{N-k} \right) \\
&= \frac{AN ( \cos(\phi) + i \sin(\phi) )}{2} \, \bs{e}_k + \frac{AN (\cos(\phi) - i\sin(\phi))}{2} \, \bs{e}_{N-k} \\
&= \frac{AN}{2} e^{i \phi} \, \bs{e}_k + \frac{AN}{2} e^{-i \phi} \, \bs{e}_{N-k}
\end{align*}
\end{proof}
\end{proposition}

\begin{definition}
The {\bf magnitude stemplot} of a complex vector $\bs{y} \in \mathbb{C}^N$ is the plot of the magnitude $| \bs{y}[n] |$ versus the index $n$. The {\bf angle stemplot} of $\bs{y}$ is the plot of the argument $\angle \bs{y}[n]$ versus the index $n$.
\end{definition}

\begin{example}
Let $N=8$ and compute the DFT of the sinusoid $\bs{x} = \sin(2 \pi \bs{t})$. Since $\bs{x} = \cos(2 \pi \bs{t} - \pi/2)$ we have $A=1$, $k=1$ and $\phi = -\pi/2$, and so
$$
\bs{y} = \mathrm{DFT}(\bs{x}) = -4i \bs{e}_1 + 4i \bs{e}_7 = \begin{bmatrix} 0 & -4i & 0 & 0 & 0 & 0 & 0 & 4i \end{bmatrix}^T
$$
The magnitude stemplot of $\bs{y} = \mathrm{DFT}(\bs{x})$ is given by
\begin{center}
\begin{tikzpicture}
\draw [->] (0,0) -- (8,0) node[below right] {$n$}; \draw [->] (0,0) -- (0,1.5) node[above left] {$| \bs{y}[n] |$};
\draw (0.1,1) -- (-0.1,1) node[left] {4}; \draw (0.1,0.5) -- (-0.1,0.5) node[left] {2};
\foreach \x/\y in { 0/0 , 1/1 , 2/0 , 3/0 , 4/0 , 5/0 , 6/0 , 7/1 } { \filldraw (\x,0) -- (\x,\y) circle[radius=2pt]; };
\foreach \k in {0,...,7} { \draw (\k,0) node[below right] {$\k$}; };
\end{tikzpicture}
\end{center}
The angle stemplot of $\bs{y} = \mathrm{DFT}(\bs{x})$ is given by
\begin{center}
\begin{tikzpicture}
\draw [->] (0,0) -- (8,0) node[below right] {$n$}; \draw [->] (0,-1.5) -- (0,1.5) node[above left] {$\angle \bs{y}[n] $};
\draw (0.1,1) -- (-0.1,1) node[left] {$\pi/2$}; \draw (0.1,-1) -- (-0.1,-1) node[left] {$-\pi/2$};
\foreach \x/\y in { 0/0 , 1/-1 , 2/0 , 3/0 , 4/0 , 5/0 , 6/0 , 7/1 } { \filldraw (\x,0) -- (\x,\y) circle[radius=2pt]; };
\foreach \k in {0,...,7} { \draw (\k,0) node[below right] {$\k$}; };
\end{tikzpicture}
\end{center}

\end{example}

\begin{example}
Let $N=8$ and let
$$
\bs{x} = \begin{bmatrix} 1 & 1 & 1 & 1 & -1 & -1 & -1 & -1 \end{bmatrix}^T
$$
Sketch the signal
\begin{center}
\begin{tikzpicture}
\draw [->] (0,0) -- (8,0) node[below right] {$n$}; \draw [->] (0,-1.5) -- (0,1.5) node[above left] {$\bs{x}[n]$};
\draw (0.1,1) -- (-0.1,1) node[left] {1}; \draw (0.1,-1) -- (-0.1,-1) node[left] {$-1$};
\foreach \x/\y in { 0/1 , 1/1 , 2/1 , 3/1 , 4/-1 , 5/-1 , 6/-1 , 7/-1 } { \filldraw (\x,0) -- (\x,\y) circle[radius=2pt]; };
\foreach \k in {0,...,7} { \draw (\k,0) node[below right] {$\k$}; };
\end{tikzpicture}
\end{center}
What frequencies occur in this signal? Compute the DFT
$$
\bs{y} = \mathrm{DFT}(\bs{x}) = \begin{bmatrix} 0 & 2 - 2(\sqrt{2} + 1)i & 0 & 2 - 2(\sqrt{2} - 1)i & 0 & 2 + 2(\sqrt{2} - 1)i & 2 + 2(\sqrt{2} + 1)i  \end{bmatrix}^T
$$
The magnitude stemplot of $\bs{y}$ is given by
\begin{center}
\begin{tikzpicture}
\draw [->] (0,0) -- (8,0) node[below right] {$n$}; \draw [->] (0,0) -- (0,2.5) node[above left] {$| \bs{y}[n] |$};
\draw (0.1,2) -- (-0.1,2) node[left] {6}; \draw (0.1,1) -- (-0.1,1) node[left] {3};
\foreach \x/\y in { 0/0 , 1/1.742 , 2/0 , 3/0.7216 , 4/0 , 5/0.7216 , 6/0 , 7/1.742 } { \filldraw (\x,0) -- (\x,\y) circle[radius=2pt]; };
\foreach \k in {0,...,7} { \draw (\k,0) node[below right] {$\k$}; };
\end{tikzpicture}
\end{center}
The angle stemplot of $\bs{y}$ is given by
\begin{center}
\begin{tikzpicture}
\draw [->] (0,0) -- (8,0) node[below right] {$n$}; \draw [->] (0,-2.5) -- (0,2.5) node[above left] {$\angle \bs{y}[n] $};
\draw (0.1,2) -- (-0.1,2) node[left] {$\pi/2$}; \draw (0.1,-2) -- (-0.1,-2) node[left] {$-\pi/2$};
\draw (0.1,1) -- (-0.1,1) node[left] {$\pi/4$}; \draw (0.1,-1) -- (-0.1,-1) node[left] {$-\pi/4$};
\foreach \x/\y in { 0/0 , 1/-1.5 , 2/0 , 3/-0.5 , 4/0 , 5/0.5 , 6/0 , 7/1.5 } { \filldraw (\x,0) -- (\x,\y) circle[radius=2pt]; };
\foreach \k in {0,...,7} { \draw (\k,0) node[below right] {$\k$}; };
\end{tikzpicture}
\end{center}
Therefore we may rewrite the signal $\bs{x}$ as a sum of sinusoids
$$
\bs{x} = A_1 \cos(2 \pi \bs{t} + \phi_1) + A_3 \cos(6 \pi \bs{t} + \phi_3)
$$
where
$$
A_1 = 2 \sqrt{4 + 2\sqrt{2}}
\hspace{5mm}
\phi_1 = -\frac{3 \pi}{8}
\hspace{5mm}
A_3 = 2 \sqrt{4 - 2\sqrt{2}}
\hspace{5mm}
\phi_3 = -\frac{\pi}{8}
$$
\end{example}

