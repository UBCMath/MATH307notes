\section{LU Decomposition with Partial Pivoting}

\begin{bigidea}
For any matrix $A$, Gaussian elimination with partial pivoting computes a decomposition $A = PLU$ where $P$ is a permutation matrix, $L$ is unit lower triangular and $U$ is upper triangular.
\end{bigidea}

\begin{note}
At step $k$ in the LU decomposition in the previous section, we form the atomic lower triangular matrix $L_k$ with entries
$$
\ell_{i,k} = - \frac{a_{i,k}^{(k)}}{a_{k,k}^{(k)}}
$$
This is a problem if $a_{k,k}^{(k)}$ very small or zero! Therefore in general we need to use pivoting. In other words, we interchange rows as necessary at each step. This leads to
$$
L_{n-1}P_{n-1} \cdots L_2 P_2 L_1 P_1 A = U
\hspace{5mm}
\Rightarrow
\hspace{5mm}
A = P_1^{-1}L_1^{-1} P_2^{-1}L_2^{-1} \cdots P_{n-1}^{-1} L_{n-1}^{-1} U
$$
We need to figure out how to rearrange the matrices into the form $A = PLU$.
\end{note}

\begin{proposition}
Let $L_k$ be an atomic lower triangular matrix with nonzero entries below the diagonal in column $k$. Let $P$ be an elementary permutation matrix which switches rows $i$ and $j$ for $i>k$ and $j>k$. Then $P L_k P = \tilde{L}_k$ where $\tilde{L}_k$ is simply $L_k$ but with entries $\ell_{i,k}$ and $\ell_{j,k}$ switched.

\begin{proof}
Write 
$L_k$ in block form
$$
L_k =
\begin{bmatrix}
1 & & & & & \\
& \ddots & & & & \\
& & 1 & & & \\
& & \ell_{k+1,k}& 1 & & \\
& & \vdots & & \ddots & \\
& & \ell_{m,k} & & & 1 \\
\end{bmatrix}
=
\left[ \begin{array}{c|c} I_k & 0 \\ \hline L_* & I_{m-k} \end{array} \right]
\hspace{10mm}
L_* = \left[ \begin{array}{cccc}
0 & \cdots & 0 & \ell_{k+1,k} \\
\vdots & & \vdots & \vdots \\
0 & \cdots & 0 & \ell_{m,k}
\end{array} \right]
$$
where $I_k$ and $I_{m-k}$ are the identity matrices of size $k$ and $m-k$ respectively. Since $P$ be an elementary permutation matrix which switches rows $i$ and $j$ for $i>k$ and $j>k$, write in block form
$$
\left[ \begin{array}{c|c} I_k & 0 \\ \hline 0 & P_* \end{array} \right]
$$
Since $P$ is an elementary permutation matrix which only switches 2 rows, we have $P^2 = I_m$ and also $P_*^2 = I_{m-k}$. Finally, using block matrices, we compute
\begin{align*}
P L_k P
&=
\left[ \begin{array}{c|c} I_k & 0 \\ \hline 0 & P_* \end{array} \right]
\left[ \begin{array}{c|c} I_k & 0 \\ \hline L_* & I_{m-k} \end{array} \right]
\left[ \begin{array}{c|c} I_k & 0 \\ \hline 0 & P_* \end{array} \right] \\
&=
\left[ \begin{array}{c|c} I_k & 0 \\ \hline 0 & P_* \end{array} \right]
\left[ \begin{array}{c|c} I_k & 0 \\ \hline L_* & P_* \end{array} \right] \\
&=
\left[ \begin{array}{c|c} I_k & 0 \\ \hline P_*L_* & P_*^2 \end{array} \right] \\
&=
\left[ \begin{array}{c|c} I_k & 0 \\ \hline P_*L_* & I_{m-k} \end{array} \right]
\end{align*}
The result is $P L_k P = \tilde{L}_k$ where $\tilde{L}_k$ is simply $L_k$ but with $\ell_{i,k}$ and $\ell_{j,k}$ switched.
\end{proof}
\end{proposition}

\begin{example}
Consider the matrices
$$
P =
\begin{bmatrix}
1 & 0 & 0 & 0 \\
0 & 0 & 0 & 1 \\
0 & 0 & 1 & 0 \\
0 & 1 & 0 & 0
\end{bmatrix}
\hspace{10mm}
L =
\begin{bmatrix}
1 & 0 & 0 & 0 \\
a & 1 & 0 & 0 \\
b & 0 & 1 & 0 \\
c & 0 & 0 & 1
\end{bmatrix}
$$
Since $P$ switches rows 2 and 4, we find
$$
P L P = \begin{bmatrix}
1 & 0 & 0 & 0 \\
c & 1 & 0 & 0 \\
b & 0 & 1 & 0 \\
a & 0 & 0 & 1
\end{bmatrix}
$$
\end{example}

\begin{theorem}
For any matrix $A$, there exists a permutation matrix $P$, unit lower triangular matrix $L$ and upper triangular matrix $U$ such that
$$
A = PLU
$$
This is the {\bf LU decomposition with partial pivoting} \cite[p.72]{MH}. The algorithm is:
\begin{enumerate}
\item[] Let $m$ be the number of rows of $A$ and let $n$ be the number of columns.
\item[] Let $A_1 = A$ and use notation $A_k = \left[ a_{i,j}^{(k)} \right]$ and $\tilde{A}_k = \left[ \tilde{a}_{i,j}^{(k)} \right]$.
\item[] {\bf for} $k$ from 1 to $n-1$:
\begin{enumerate}
\item[] Find index $p$ with maximum value $\left| a_{p,k}^{(k)} \right|$ in column $k$ of $A_k$ below diagonal
\item[] (or if all entries in column $k$ below diagonal are zero, move to the next column).
\item[] Let $P_k$ such that $P_k A_k = \tilde{A}_k$ switches rows $p$ and $k$.
\item[] Let $L_k$ such that $\ds \ell_{i,k} = - \frac{\tilde{a}^{(k)}_{i,k}}{\tilde{a}^{(k)}_{k,k}}$ and compute $L_k \tilde{A}_k = A_{k+1}$.
\end{enumerate}
\item[] {\bf end}
\item[] $A = PLU$ where $P=P_1 \cdots P_{n-1}$, $U = A_n$ and $L = \tilde{L}_1^{-1} \cdots \tilde{L}_{n-1}^{-1}$ where each $\tilde{L}_k^{-1}$ is given by $L_k^{-1}$ with entries permuted in order by $P_{k'}$ for each $k' > k$.
\end{enumerate}
\end{theorem}

\begin{note}
Pivoting ensures all the entries of $L$ below the diagonal satisfy $| \ell_{i,j} | \leq 1$ ($i > j$).
\end{note}

\begin{example}
Compute the LU decomposition with partial pivoting of
$$
A = \begin{bmatrix} 1 & 1 & 1 \\ 1 & 1 & 2 \\ 0 & 1 & 1 \end{bmatrix}
$$
In the first column, we see the maximum value $| a_{i,1}|$ is 1 and we do not need to pivot therefore $P_1 = I$ and $\tilde{A}_1 = A_1 = A$. Compute:
$$
L_1 \tilde{A}_1 =
\left[ \begin{array}{rrr} 1 & 0 & 0 \\ -1 & 1 & 0 \\ 0 & 0 & 1 \end{array} \right]
\left[ \begin{array}{rrr} 1 & 1 & 1 \\ 1 & 1 & 2 \\ 0 & 1 & 1 \end{array} \right]
=
\left[ \begin{array}{rrr} 1 & 1 & 1 \\ 0 & 0 & 1 \\ 0 & 1 & 1 \end{array} \right]
$$
In the second column, we see $a_{2,2}^{(2)} = 0$ and $a_{3,2}^{(2)} = 1$ therefore we must swap rows
$$
P_2 A_2 = \left[ \begin{array}{rrr} 1 & 1 & 1 \\ 0 & 1 & 1 \\ 0 & 0 & 1 \end{array} \right]
$$
This is already upper triangular and so $L_2 = I$ in the last step. Therefore
$$
L_2 P_2 L_1 P_1 A = U \ \ \Rightarrow \ \ A = P_1^{-1} L_1^{-1} P_2^{-1}L_2^{-1} U = L_1^{-1} P_2^{-1} U
$$
Using the proposition above we find $ L_1^{-1} P_2^{-1} =   P_2^{-1} \tilde{L}_1^{-1}$ where
$$
\tilde{L}_1^{-1} = \left[ \begin{array}{rrr} 1 & 0 & 0 \\ 0 & 1 & 0 \\ 1 & 0 & 1 \end{array} \right]
$$
Finally, we have $A = PLU$ where
$$
P = \begin{bmatrix} 1 & 0 & 0 \\ 0 & 0 & 1 \\ 0 & 1 & 0 \end{bmatrix}
\hspace{5mm}
L = \begin{bmatrix} 1 & 0 & 0 \\ 0 & 1 & 0 \\ 1 & 0 & 1 \end{bmatrix}
\hspace{5mm}
U = \begin{bmatrix} 1 & 1 & 1 \\ 0 & 1 & 1 \\ 0 & 0 & 1 \end{bmatrix}
$$
\end{example}

\begin{example}
Suppose the LU decomposition with partial pivoting applied to $A$ yields matrices
$$
P_1 = \begin{bmatrix} 0 & 0 & 0 & 1 \\ 0 & 1 & 0 & 0 \\ 0 & 0 & 1 & 0 \\ 1 & 0 & 0 & 0 \end{bmatrix}
\hspace{5mm}
P_2 = \begin{bmatrix} 1 & 0 & 0 & 0 \\ 0 & 0 & 1 & 0 \\ 0 & 1 & 0 & 0 \\ 0 & 0 & 0 & 1 \end{bmatrix}
\hspace{5mm}
P_3 = \begin{bmatrix} 1 & 0 & 0 & 0 \\ 0 & 1 & 0 & 0 \\ 0 & 0 & 0 & 1 \\ 0 & 0 & 1 & 0 \end{bmatrix}
$$
$$
L_1 = \left[ \begin{array}{cccc} 1 & 0 & 0 & 0 \\ 0.2 & 1 & 0 & 0 \\ -0.1 & 0 & 1 & 0 \\ 0.5 & 0 & 0 & 1 \end{array} \right]
\hspace{5mm}
L_2 = \left[ \begin{array}{cccc} 1 & 0 & 0 & 0 \\ 0 & 1 & 0 & 0 \\ 0 & 0.7 & 1 & 0 \\ 0 & -0.1 & 0 & 1 \end{array} \right]
\hspace{5mm}
L_3 = \left[ \begin{array}{cccc} 1 & 0 & 0 & 0 \\ 0 & 1 & 0 & 0 \\ 0 & 0 & 1 & 0 \\ 0 & 0 & 0.1 & 1 \end{array} \right]
$$
We rearrange the matrices
\begin{align*}
& P_1^{-1} L_1^{-1} P_2^{-1} L_2^{-1} P_3^{-1} L_3^{-1} \\
&=
P_1
\left[ \begin{array}{cccc} 1 & 0 & 0 & 0 \\ -0.2 & 1 & 0 & 0 \\ 0.1 & 0 & 1 & 0 \\ -0.5 & 0 & 0 & 1 \end{array} \right]
P_2
\left[ \begin{array}{cccc} 1 & 0 & 0 & 0 \\ 0 & 1 & 0 & 0 \\ 0 & -0.7 & 1 & 0 \\ 0 & 0.1 & 0 & 1 \end{array} \right]
P_3
\left[ \begin{array}{cccc} 1 & 0 & 0 & 0 \\ 0 & 1 & 0 & 0 \\ 0 & 0 & 1 & 0 \\ 0 & 0 & -0.1 & 1 \end{array} \right] \\
&=
P_1 P_2
\left[ \begin{array}{cccc} 1 & 0 & 0 & 0 \\ 0.1 & 1 & 0 & 0 \\ -0.2 & 0 & 1 & 0 \\ -0.5 & 0 & 0 & 1 \end{array} \right]
P_3
\left[ \begin{array}{cccc} 1 & 0 & 0 & 0 \\ 0 & 1 & 0 & 0 \\ 0 & 0.1 & 1 & 0 \\ 0 & -0.7 & 0 & 1 \end{array} \right]
\left[ \begin{array}{cccc} 1 & 0 & 0 & 0 \\ 0 & 1 & 0 & 0 \\ 0 & 0 & 1 & 0 \\ 0 & 0 & -0.1 & 1 \end{array} \right] \\
&=
P_1 P_2 P_3
\left[ \begin{array}{cccc} 1 & 0 & 0 & 0 \\ 0.1 & 1 & 0 & 0 \\ -0.5 & 0 & 1 & 0 \\ -0.2 & 0 & 0 & 1 \end{array} \right]
\left[ \begin{array}{cccc} 1 & 0 & 0 & 0 \\ 0 & 1 & 0 & 0 \\ 0 & 0.1 & 1 & 0 \\ 0 & -0.7 & 0 & 1 \end{array} \right]
\left[ \begin{array}{cccc} 1 & 0 & 0 & 0 \\ 0 & 1 & 0 & 0 \\ 0 & 0 & 1 & 0 \\ 0 & 0 & -0.1 & 1 \end{array} \right] \\
&=
P_1 P_2 P_3
\left[ \begin{array}{cccc} 1 & 0 & 0 & 0 \\ 0.1 & 1 & 0 & 0 \\ -0.5 & 0.1 & 1 & 0 \\ -0.2 & -0.7 & -0.1 & 1 \end{array} \right]
\end{align*}
Therefore we have found $P$ and $L$ in $A = PLU$ where
$$
P = P_1 P_2 P_3 = \begin{bmatrix} 0 & 0 & 1 & 0 \\ 0 & 0 & 0 & 1 \\ 0 & 1 & 0 & 0 \\ 1 & 0 & 0 & 0 \end{bmatrix}
\hspace{5mm}
L = \left[ \begin{array}{cccc} 1 & 0 & 0 & 0 \\ 0.1 & 1 & 0 & 0 \\ -0.5 & 0.1 & 1 & 0 \\ -0.2 & -0.7 & -0.1 & 1 \end{array} \right]
$$
\end{example}

\begin{note}
Mathematical software such as MATLAB {\tt linsolve} (see \href{https://www.mathworks.com/help/matlab/ref/linsolve.html}{MATLAB documentation}) and SciPy {\tt scipy.linalg.solve} (see \href{https://docs.scipy.org/doc/scipy-0.14.0/reference/generated/scipy.linalg.solve.html}{SciPy documentation}) compute solutions of linear systems by LU decomposition with partial pivoting.
\end{note}