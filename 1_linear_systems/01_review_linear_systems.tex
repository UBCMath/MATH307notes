\section{Review: Linear Systems}

\begin{bigidea}
Solve a linear system of equations $A \bs{x} = \bs{b}$ by reducing the augmented matrix $[ \, A \ \bs{b} \, ]$ to row-echelon form via Gaussian elimination.
\end{bigidea}

\begin{definition}
A {\bf linear system of equations} \cite[p.1]{KN} is a collection of equations of the form
$$
\begin{array}{ccccccccc}
a_{1,1} x_1 & + & a_{1,2} x_2 & + & \cdots & + & a_{1,n} x_n & = & b_1 \\
a_{2,1} x_1 & + & a_{2,2} x_2 & + & \cdots & + & a_{2,n} x_n & = & b_2 \\
& & & & & & & \vdots & \\
a_{m,1} x_1 & + & a_{m,2} x_2 & + & \cdots & + & a_{m,n} x_n & = & b_m
\end{array}
$$
where the coefficients $a_{i,j}$, $b_i$ are known constants. In matrix notation, we have $A \bs{x} = \bs{b}$ where
$$
A = 
\begin{bmatrix}
a_{1,1} & a_{1,2} & \cdots & a_{1,n} \\
a_{2,1} & a_{2,2} & \cdots & a_{2,n} \\
\vdots & \vdots & \ddots & \vdots \\
a_{m,1} & a_{m,2} & \cdots & a_{m,n}
\end{bmatrix}
\hspace{10mm}
\bs{x} = \begin{bmatrix} x_1 \\ x_2 \\ \vdots \\ x_n \end{bmatrix}
\hspace{10mm}
\bs{b} = \begin{bmatrix} b_1 \\ b_2 \\ \vdots \\ b_m \end{bmatrix}
$$
\end{definition}

\begin{definition}
The following are called {\bf elementary row operations} \cite[p.5]{KN}:
\begin{enumerate}
\item Interchange two rows.
\item Multiply one row by a nonzero number.
\item Add a multiple of one row to a different row.
\end{enumerate}
\end{definition}

\begin{definition}
A matrix is in {\bf row-echelon form} \cite[p.10]{KN} if:
\begin{enumerate}
\item All zero rows are at the bottom.
\item The first nonzero entry in a row (from the left) is 1.
\item The first nonzero entry in a row (from the left) is located to the right of the first nonzero entry in every row above.
\end{enumerate}
For example:
$$
\begin{bmatrix} 1 & * & * \\ 0 & 1 & * \\ 0 & 0 & 1 \end{bmatrix} \hspace{5mm}
\begin{bmatrix} 1 & * & * & * & * \\ 0 & 1 & * & * & * \\ 0 & 0 & 1 & * & * \\ 0 & 0 & 0 & 0 & 0 \end{bmatrix} \hspace{5mm}
\begin{bmatrix} 1 & * & * & * & * \\ 0 & 0 & 1 & * & * \\ 0 & 0 & 0 & 0 & 1 \\ 0 & 0 & 0 & 0 & 0 \end{bmatrix} \hspace{5mm}
\begin{bmatrix}  1 & * \\ 0 & 0 \end{bmatrix}
$$
\end{definition}

\begin{theorem}
Every matrix can be transformed to row-echelon form by a sequence of elementary row operations via the Gaussian elimination algorithm \cite[p.11]{KN}.
\end{theorem}

\begin{example}
Find all solutions of $A \bs{x} = \bs{b}$ for:
\begin{enumerate}
\item $A = \begin{bmatrix} 1 & 1 & 1 \\ 1 & 0 & 1 \\ 2 & 5 & 2 \end{bmatrix}$ \hspace{3mm} $\bs{b} = \begin{bmatrix} 2 \\ 1 \\ 7 \end{bmatrix}$
\item $A = \left[ \begin{array}{rrrr} 1 & -1 & \phantom{+}1 & -2 \\ -1 & 1 & 1 & 1 \\ -1 & 2 & 3 & 1 \\ 1 & -1 & 2 & 1 \end{array} \right]$ \hspace{3mm} $\bs{b} = \left[ \begin{array}{r} 1 \\ -1 \\ 2 \\ 1 \end{array} \right]$
\item $A = \begin{bmatrix} 1 & 1 & 2 & 1 \\ 1 & 0 & 1 & 1 \\ 0 & 1 & 1 & 0 \end{bmatrix}$ \hspace{3mm} $\bs{b} = \begin{bmatrix} 1 \\ 1 \\ 2 \end{bmatrix}$
\end{enumerate}
Implement Gaussian elimination for each example:
\begin{enumerate}
\item $\left[ \begin{array}{rrr|r} 1 & 1 & 1 & 2 \\ 1 & 0 & 1 & 1 \\ 2 & 5 & 2 & 7 \end{array} \right] \ \longrightarrow \ \left[ \begin{array}{rrr|r} 1 & 1 & 1 & 2 \\ 0 & 1 & 0 & 1 \\ 0 & 0 & 0 & 0 \end{array} \right] \Rightarrow \ \ \bs{x} = \begin{bmatrix} 1-t \\ 1 \\ t \end{bmatrix} \ , \ \ t \in \mathbb{R}$
\item $\left[ \begin{array}{rrrr|r} 1 & -1 & \phantom{+}1 & -2 & 1 \\ -1 & 1 & 1 & 1 & -1 \\ -1 & 2 & 3 & 1 & 2 \\ 1 & -1 & 2 & 1 & 1 \end{array} \right] \longrightarrow \left[ \begin{array}{rrrr|r} 1 & -1 & \phantom{+}1 & -2 & \phantom{+}1 \\ 0 & 1 & 4 & -1 & 3 \\ 0 & 0 & 1 & 3 & 0 \\ 0 & 0 & 0 & 1 & 0 \end{array} \right]  \Rightarrow \ \ \bs{x} = \begin{bmatrix} 4 \\ 3 \\ 0 \\ 0 \end{bmatrix}$
\item $\left[ \begin{array}{rrrr|r} 1 & 1 & 2 & 1 & 1 \\ 1 & 0 & 1 & 1 & 1 \\ 0 & 1 & 1 & 0 & 2 \end{array} \right] \longrightarrow \left[ \begin{array}{rrrr|r} 1 & 1 & 2 & 1 & 1 \\ 0 & 1 & 1 & 0 & 0 \\ 0 & 0 & 0 & 0 & 1 \end{array} \right] \ \Rightarrow \ \ $ No solution
\end{enumerate}
\end{example}

\begin{definition}
The {\bf rank} of a matrix $A$ \cite[p.16]{KN} is the number of nonzero rows in the row-echelon form of $A$.
\end{definition}

\begin{example} Let $A$ be $m \times n$ matrix with $\mathrm{rank}(A) = r$. Describe when the linear system $A \bs{x} = \bs{b}$ has: a unique solution, infinitely many solutions, or no solutions. \\

The system $A \bs{x} = \bs{b}$ is inconsistent (ie.~no solution) if $\mathrm{rank}(A) < \mathrm{rank}([ \, A \ \bs{b} \, ])$. That is, the row-echelon form of the augmented matrix $[ \, A \ \bs{b} \, ]$ has a row of the form
$$
\begin{array}{rrrr|r} 0 & 0 & \cdots & 0 & 1 \end{array}
$$
which implies $0 = 1$. The system has a unique solution when $\mathrm{rank}(A) = \mathrm{rank}([ \, A \ \bs{b} \, ])$ and $\mathrm{rank}(A) = n$. That is, the rank is equal to the number of variables in the system. Finally, the system has infinitely many solutions when $\mathrm{rank}(A) = \mathrm{rank}([ \, A \ \bs{b} \, ])$ and $\mathrm{rank}(A) < n$.
\end{example}